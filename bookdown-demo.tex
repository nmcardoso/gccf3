% Options for packages loaded elsewhere
\PassOptionsToPackage{unicode}{hyperref}
\PassOptionsToPackage{hyphens}{url}
%
\documentclass[
]{book}
\usepackage{lmodern}
\usepackage{amssymb,amsmath}
\usepackage{ifxetex,ifluatex}
\ifnum 0\ifxetex 1\fi\ifluatex 1\fi=0 % if pdftex
  \usepackage[T1]{fontenc}
  \usepackage[utf8]{inputenc}
  \usepackage{textcomp} % provide euro and other symbols
\else % if luatex or xetex
  \usepackage{unicode-math}
  \defaultfontfeatures{Scale=MatchLowercase}
  \defaultfontfeatures[\rmfamily]{Ligatures=TeX,Scale=1}
\fi
% Use upquote if available, for straight quotes in verbatim environments
\IfFileExists{upquote.sty}{\usepackage{upquote}}{}
\IfFileExists{microtype.sty}{% use microtype if available
  \usepackage[]{microtype}
  \UseMicrotypeSet[protrusion]{basicmath} % disable protrusion for tt fonts
}{}
\makeatletter
\@ifundefined{KOMAClassName}{% if non-KOMA class
  \IfFileExists{parskip.sty}{%
    \usepackage{parskip}
  }{% else
    \setlength{\parindent}{0pt}
    \setlength{\parskip}{6pt plus 2pt minus 1pt}}
}{% if KOMA class
  \KOMAoptions{parskip=half}}
\makeatother
\usepackage{xcolor}
\IfFileExists{xurl.sty}{\usepackage{xurl}}{} % add URL line breaks if available
\IfFileExists{bookmark.sty}{\usepackage{bookmark}}{\usepackage{hyperref}}
\hypersetup{
  pdftitle={Anotações GCFA3},
  pdfauthor={Natanael Magalhães Cardoso},
  hidelinks,
  pdfcreator={LaTeX via pandoc}}
\urlstyle{same} % disable monospaced font for URLs
\usepackage{longtable,booktabs}
% Correct order of tables after \paragraph or \subparagraph
\usepackage{etoolbox}
\makeatletter
\patchcmd\longtable{\par}{\if@noskipsec\mbox{}\fi\par}{}{}
\makeatother
% Allow footnotes in longtable head/foot
\IfFileExists{footnotehyper.sty}{\usepackage{footnotehyper}}{\usepackage{footnote}}
\makesavenoteenv{longtable}
\usepackage{graphicx,grffile}
\makeatletter
\def\maxwidth{\ifdim\Gin@nat@width>\linewidth\linewidth\else\Gin@nat@width\fi}
\def\maxheight{\ifdim\Gin@nat@height>\textheight\textheight\else\Gin@nat@height\fi}
\makeatother
% Scale images if necessary, so that they will not overflow the page
% margins by default, and it is still possible to overwrite the defaults
% using explicit options in \includegraphics[width, height, ...]{}
\setkeys{Gin}{width=\maxwidth,height=\maxheight,keepaspectratio}
% Set default figure placement to htbp
\makeatletter
\def\fps@figure{htbp}
\makeatother
\setlength{\emergencystretch}{3em} % prevent overfull lines
\providecommand{\tightlist}{%
  \setlength{\itemsep}{0pt}\setlength{\parskip}{0pt}}
\setcounter{secnumdepth}{5}
\usepackage{booktabs}
\usepackage{amsthm}
\makeatletter
\def\thm@space@setup{%
  \thm@preskip=8pt plus 2pt minus 4pt
  \thm@postskip=\thm@preskip
}
\makeatother
\usepackage[]{natbib}
\bibliographystyle{apalike}

\title{Anotações GCFA3}
\author{Natanael Magalhães Cardoso}
\date{2022-10-13}

\begin{document}
\maketitle

{
\setcounter{tocdepth}{1}
\tableofcontents
}
\hypertarget{sobre}{%
\chapter{Sobre}\label{sobre}}

Este caderno possui anotações de alguns tópicos do programa de
treinamento online Google Cloud Computing Foundations 3.

\hypertarget{cloud-computing-fundamentals}{%
\chapter{Cloud Computing Fundamentals}\label{cloud-computing-fundamentals}}

\hypertarget{cloud-computing}{%
\section{Cloud Computing}\label{cloud-computing}}

A computação em nuvem é uma continuação de uma mudança de longo prazo na
forma como a computação e os recursos são gerenciados. É a continuação de um
modelo na qual o cliente aluga uma infraestrutura de computação que é gerenciada
por proficionais dedicados. A Equinix e a CenturyLink são as duas maiores
provedoras de data centers nos EUA.

\hypertarget{primeira-onda}{%
\subsection{Primeira Onda}\label{primeira-onda}}

O conceito da computação em nuvem começou com ``colocação'', que tradicionalmente
não é considerada como computação em núvem, mas foi o início do processo de
transferência da infraestrutura de TI para fora da organização. As orgamizações
economizavam dinheiro como a ``colocação'' pois não precisavam construir datacenters
e os serviços relacionados. O provedor de ``colocação'' alugava tudo para a organização.

\hypertarget{segunda-onda}{%
\subsection{Segunda Onda}\label{segunda-onda}}

Com a computação em núvem surgiram os data centers virtualizados, as máquinas
virtuais e as API's. A virtualização oferece elasticidade, uma vez que o
cliente automiza a aquisição da infrawstrutura ao invés de comprar hardware.
Com a virtualização, a infraestrutura ainda é mantida. O ambiente ainda é
controlado e configurado pelo cliente. Ela é como um data center local, mas
com a diferença de que o hardware está em outro local.

\hypertarget{terceira-onda}{%
\subsection{Terceira Onda}\label{terceira-onda}}

A onda seguinte da computação em núvem foi a núvem elástica totalmente automatizada.
Ao invés do cliente manter a infraestrutura, passou a ter serviços automatizados.
Em uma ambiente totalmente automatizado, os desenvolvedores não pensam em máquinas
individuais. O serviço de provisiona e configura automaticamente a infraestrutura
necessária para executar os aplicativos.

\hypertarget{timeline}{%
\subsection{Timeline}\label{timeline}}

\begin{itemize}
\item
  \textbf{1980:} First Wave: Server on premises

  You own everything. It's yours to manage.
\item
  \textbf{2000:} Second Wave: Data centers

  You pay for the hardware but rent the space. Still yours to manage.
\item
  \textbf{2006:} First Generation Cloud: Virtualized data centers

  You rent hardware and space, but still control and configure virtual machines.
  Pay for what you provision.
\item
  \textbf{2009:} Third Wave: Managed Service

  Completely elastic storage, processing, and machine learning so that you
  can invest your energy in great apps. Pay for what you use.
\end{itemize}

\hypertarget{iaas-paas-saas}{%
\section{IaaS, PaaS, SaaS}\label{iaas-paas-saas}}

Este tópico mostra as diferenças principais entre infraestrutura como serviço,
plataforma como serviço e software como serviço.

\hypertarget{iaas}{%
\subsection{IaaS}\label{iaas}}

O serviço fornece a arquitetura subjacente para você executar os servidores.
CPU, memória, armazenamento e rede são disponibilizados como serviço, mas
o usuário precisa gerenciar o sistema operacional e a aplicação.

\hypertarget{paas}{%
\subsection{PaaS}\label{paas}}

Todo ambiente será gerenciado pera o usuário, que, portanto, só precisará
gerenciar seus aplicativos. A camada do sistema operacional é gerenciada como
parte do serviço

\hypertarget{saas}{%
\subsection{SaaS}\label{saas}}

A infraestrutura, a plataforma e o software são gerenciados pera o usuário.
Você só precisa colocar seus dados no sistema. O SAP e Salesforce são exemplos
comerciais de SaaS.

\hypertarget{produtos-gcp}{%
\subsection{Produtos GCP}\label{produtos-gcp}}

\begin{itemize}
\tightlist
\item
  Compute Engine: IaaS (pay for what you allocate)
\item
  App Engine: PaaS (pay for what you use)
\item
  Managed Services: Automated elastic resources
\item
  Google Kubernetes Engine: Hybrid
\item
  Cloud Functions: Serverless logic
\end{itemize}

\hypertarget{google-cloud-architecture}{%
\section{Google Cloud Architecture}\label{google-cloud-architecture}}

.

\hypertarget{networking-and-security}{%
\chapter{Networking and Security}\label{networking-and-security}}

\hypertarget{nuxfavens-huxedbridas}{%
\section{Núvens Híbridas}\label{nuxfavens-huxedbridas}}

\begin{itemize}
\tightlist
\item
  Útil para conexões de baixo volume \emph{(pois os dados passam pela internet
  pública)}
\item
  Suporta:

  \begin{itemize}
  \tightlist
  \item
    Site-to-site VPN
  \item
    Rotas estáticas ou dinâmicas \emph{(Cloud Router)}
  \item
    Criptografias IKEv1 e IKEv2
  \end{itemize}
\item
  Não é compatível com conectividade cliente -\textgreater{} gateway. Isto é, não é possível
  que computadores clientes disquem para a VPN usando um software cliente.
\end{itemize}

\emph{Cloud VPN} conecta redes locais (on-premisses) a redes privadas virtuais
\emph{(VPC)} através de túnel VPN IPsec. O tráfico trocado entre as duas redes
é criptografado por um \emph{Gateway} da \emph{VPN} e descriptografado por outro
\emph{Gateway} para proteger os dados que passam pela Internet pública.

\hypertarget{interconexuxe3o}{%
\section{Interconexão}\label{interconexuxe3o}}

O \emph{Cloud Interconect} oferece duas opções para estender uma rede local para uma
rede VPC:

\begin{itemize}
\tightlist
\item
  Cloud Interconnect Dedicated: forma uma conexão física direta entre a rede
  local da organização e a borda de rede do Google Cloud, de modo que seja
  possível transferir grande volume de dados entre as redes. Pode ser mais
  econômico que comprar mais largura de banda na Internet pública.
\item
  Cloud Interconnect Partner
\end{itemize}

  \bibliography{book.bib,packages.bib}

\end{document}
